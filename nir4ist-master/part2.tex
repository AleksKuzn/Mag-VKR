% вторая часть

\section{Обзор аппаратно-программного комплекса «СКАУТ»}
Обычно, жилой дом становится умным, благодаря применению в нем аппаратно-программного комплекса СКАУТ. Задача СКАУТА --- усиление безопасности, экономия ресурсов и повышение комфорта жителей. \cite{Almanah}

Функции аппаратно-программного комплекса СКАУТ можно разделить на семь, тесно связанных между собой, систем.
\begin{enumerate}
	\item Контроль доступа; 
	\item Учет ресурсов;
	\item Оповещение;
	\item Wi-Fi;
	\item Охрана технических помещений;
	\item Управление оборудованием;
	\item Видеонаблюдение.
\end{enumerate} 

Для доступа или въезда на территорию комплекса используется универсальный электронный ключ. Ключ является персональным для жителей, его копирование или подделка исключена. Все технические помещения находятся под контролем скаута. Снятие и постановка на охрану осуществляется техническим персоналом самостоятельно, с помощью мобильного телефона. Универсальный механический ключ удобен в эксплуатации и облегчает перемещение работников. 

Все потребляемые жителями ресурсы полностью учитываются и публикуются на сайте управляющей компании. Автоматизированный анализ показаний позволяет своевременно выявлять нарушении режимов отопления, неисправности приборов учета, возникающие аварии. 

Громкоговорящая система оповещения и информирования вовремя предупредит о планируемых ремонтных работах, причинах и сроках ликвидации аварийных ситуаций. Оператору управляющей компанией достаточно набрать текст сообщения и очертить на электронной карте зону оповещения. 

В чрезвычайных ситуациях диспетчер управляющей компании в ручном режиме может управлять электрооборудованием дома: шлагбаумами, электрозадвижками, системой дымоудаления, пожарными насосами.

Система видеонаблюдения, кроме охранных функций, осуществляет контроль исполнения команд диспетчера. Дает информацию о заполненности парковки на мобильные устройства. Система видеонаблюдения включена в муниципальный комплекс Безопасный город.

Используя Wi-Fi технологию, СКАУТ обеспечивает работникам управляющей компанией мобильный доступ к проектной документации, связь с диспетчером, передачу контрольной и видео информаций в аварийных случаях. 

Каждый житель цифрового района имеет доступ к данным системам СКАУТ через личный кабинет, в котором может получить видеоинформацию, организовать доступ гостей и родственников, получить информацию по коммунальным платежам. 

Умный город состоит из умных домов, а пока СКАУТ ступень к новому качеству жизни.