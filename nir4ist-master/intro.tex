% введение

В современных условиях разработка и реализация концепции Умного города остается одним из главных направлений развития городов в индустриально развитых странах. Это наиболее явно проявляется в странах, столкнувшихся с целым спектром инфраструктурных и социальных проблем. 

Умный город --- это взаимосвязанная система коммуникативных и информационных технологий с интернетом вещей (IoT), благодаря которой упрощается управление внутренними процессами города и улучшается уровень жизни населения.

Плюсы проекта Умного города заключаются в повышении уровня жизни граждан и в уменьшении издержек рабочих процессов благодаря автоматизации деятельности, не требующей применения аналитических навыков.

Удорожание тарифов на тепловую энергию, горячую и холодную воду приводит к тому, что потребители всё больше задумываются о точной и своевременной оценке количества потреблённых ресурсов. Повсеместная установка приборов учёта является сегодня одним из приоритетных направлений реформирования ЖКХ. Однако, кроме монтажа счётчика, необходимо обеспечить возможность оперативного и регулярного снятия показаний с него. Пока счётчиков мало, эту операцию можно проводить и вручную, но как только количество узлов учёта начинает исчисляться десятками и сотнями, возникает задача создания системы автоматического сбора показаний. Такая диспетчеризация позволяет не только оперативно собирать данные, но и проводить всесторонний анализ работы теплосетей (например, выявлять неисправности).

Есть большая потребность в обработке поступающих данных в реальном времени с целью выявления аварийных и предаварийных ситуаций, нарушений работоспособности счетчиков водоснабжения, электроэнергии и режимов теплоснабжения. Такие потребности возникают и у потребителей, и у эксплуатирующих организаций, и у поставщиков тепловой энергии.

Основным результатом выполнения проекта будет программное обеспечение, которое в автоматизированном режиме проводит анализ данных, выявляет ошибки и проблемные счетчики.

Задачи, решаемые в ходе работы (в соответствии с заданием на ВКР):
\begin{enumerate}
	\item Обзор проектов Умный город;
	\item Анализ существующих решений проблемы;
	\item Изучение аппаратно-программного комплекса «СКАУТ»; 
	\item Разработка системы квартирного учета и анализа потребления ресурсов;
	\item Разработка программного обеспечения системы учета ресурсов;
	\item Тестирование и оптимизация программного обеспечения;	
\end{enumerate}